% ssk_impl.tex - SSK implementation

\iffalse
Copyright (C) 2018 -- Matthew R. Wette.

Permission is granted to copy, distribute and/or modify this document
under the terms of the GNU Free Documentation License, Version 1.3 or
any later version published by the Free Software Foundation; with no
Invariant Sections, no Front-Cover Texts, and no Back-Cover Texts.  A
copy of the license is included with the distribution as @file{COPYING.DOC}.
\fi

\documentclass{article}
\usepackage{times}
\usepackage{amsmath}

\setlength{\textheight}{9in}
\setlength{\textwidth}{6.5in}
\setlength{\topmargin}{-.5in}
\setlength{\evensidemargin}{0in}
\setlength{\oddsidemargin}{0in}

\begin{document}


\section{The Reader (mdreader.py)}

The reader is a sax type xml reader.  There are a set of classes,
derived from LtWtHdlr that have two routines begEltIn and endEltIn.
In the context of a handler, new tags are passed to the begEltIn method.
If there is a handler class for this tag then the new handler is pushed
onto the handler stack and takes over.  When the new endEltIn sees the
tag's end-tag then the handler is popped off the stack and the parent
handler takes over.

The reader has a cleanup capability that should be used.  For example,
the Vertex class has arrays for incoming and outgoing transitions but
these are not populated by MD17.


\section{The Validator (anlyz.py)}

The uml2 package has a validator that can be used to catch errors as well
as enforce sublangage definitions (e.g., restrictions on state naming, or 
use of certain events, or use of certain syntax).

\section{The Translater (uml2ssk.py)}

This section is about the translator

\section{The Code Generators}

This section is about the code generation.

TODO list:
\begin{enumeration}
\item enumerate the actions 
\item fix eval routine so that it gets/puts events and state.
\end{enumeration}


\subsection{Implied Transitions}

\begin{itemize}
\item to orthogonal state: implied initial states
\item from othogonal state: ???
\end{itemize}


\subsection{Encoding the State}

The machine state is encoded as an array of leaf-state identifiers.
In the case of a machine which has no orthongonal states the array has
length one.  Each state has a unique identifier.  If a state contains
multiple regions (i.e., it is an orthogonal state), then the state
will occupy multiple elements in the array.

NOTE: we should make sure lowest identifiers are used for simple states

We could encode states by regional context, using offset into the region. 
In this case the enumerations will be kept small.  Or we could generate a 
unique id for each state in the machine.  This may potentially generate
a large number for the state.  We'd like to keep the number small (e.g., <256)
so that we can encode a state using a single byte.

CONJECTURE: offset 0 will always be nonzero

Info
\begin{itemize}
\item Within any region, only one state is active at any time.
\item For each composite state in a region, the max number of slots
  needed is bounded by the sum of the max used for each region.
\end{itemize}

The state encoding gets described here.  We need to handle parallel transitions?
\begin{itemize}
\item only simple state ids are used in the encoding
\item in a region each simple or composite state requires one slot
\item in a region each orthogonal state requires the max slots for that state
\item the maximum slots required to encode a state is equial to the max over
all regions in the state
\item (prove) each each region has a unique offset into the slot array
\item (prove) each (simple) state has a unique offset into the slot
  array, which is given by the offset of the parent region of that state
\end{itemize}

For simple states we generate an array of the state offset:
\begin{verbatim}
oset_enc[] = { 0, 2, 1, 0, 0, 1 };
\end{verbatim}
and then to determine if we are in that state we ask
\begin{verbatim}
  encoding[oset_end[state]] == state ?
\end{verbatim}

One could use unique state ids or state offsets in the region, but
state ids are probably better.

When there is a transition the deepest common region (LCA) will be changed. 
Need to look at situations where a bla bla bla

State id zero is reserved for nothing (or root state).

\begin{verbatim}
for (i = 0; i < N; i++) {
  switch(mst[i]) {
  case 1: ... break;
  case 2: ... break;
  case 255: ... break;
  }
}
now what?
\end{verbatim}

\subsection{Flattening}

It is possible to flatten statecharts.  Consider a statechart which has
no othogonal regions.  (There is code in the implementation to flatten
statecharts.  I'm not sure this is fully implemented.)


\end{document}




