%% sskguide.tex

\iffalse
Copyright (C) 2018 -- Matthew R. Wette.

Permission is granted to copy, distribute and/or modify this document
under the terms of the GNU Free Documentation License, Version 1.3 or
any later version published by the Free Software Foundation; with no
Invariant Sections, no Front-Cover Texts, and no Back-Cover Texts.  A
copy of the license is included with the distribution as @file{COPYING.DOC}.
\fi

\documentclass{article}
\usepackage{times}
\usepackage{amsmath}
\usepackage{epsfig}

\setlength{\textheight}{9in}
\setlength{\textwidth}{6.5in}
\setlength{\topmargin}{-.5in}
\setlength{\evensidemargin}{0in}
\setlength{\oddsidemargin}{0in}

\title{%
Simple Statechart Toolkit (SSK)
}
\author{Matt Wette \\ project }
\date{\large DRAFT of \noindent\today \ /\the\time}

\begin{document}
\maketitle

This document covers development of concept and implementation for 
statecharts.  The idea is to make something that is restrictive enough
to make statecharts simple to understand and implement but rich enough
to allow good design.

\section{architecture}

Contrary to other statechart architectures we should use ports (instead of
objects).  The 0 input/output ports are called ``self'' and allow us to 
send events to ourself.

Ports should have events sequences and state somehow.  I need to reconsile
this with the systems theory. Read Willems stuff on interconnections.

two output functions: one for moore (statesymbol), one for mealy (events)

\section{Concept}

We start with the elements of SSK:
\begin{itemize}
\item
state
\item
region
\item
transition
\item
event (or event labels)
\item
guard - expression resulting in boolean.  Can we constrain this
somehow?  Maybe to boolean variable or enumeration variable (mode ==
X).  More complicated expressions are those like ``x < 10''.
\item
action port:signal or var=val
\end{itemize}


\section{theory}

A statemachine is an element of a dynamic system with inputs and outputs 
including piecewise constant functions and sequences of events.  That is,
statemachine are union of Moore and Mealy.
This document covers a program, written in Python, for working
with simple statecharts.

We need to define what is in and what is out.

\begin{itemize}
\item
For each xxx
\item
Watch out for side effects of event arguments!  These need to 
be managed somehow.  (e.g., model of state)
\item
Need to decide if asynchronous or synchronous.  Can these come together?
Synchronous part can only execute do-part, read condition vars and
set status
\end{itemize}

We need to define what is in and what is out.


\begin{itemize}
\item
For each xxx
\item
Watch out for side effects of event arguments!  These need to 
be managed somehow.  (e.g., model of state)
\item
Need to decide if asynchronous or synchronous.  Can these come together?
Synchronous part can only execute do-part, read condition vars and
set status
\end{itemize}


Syntax rules:
\begin{itemize}
\item 
transitions are decorated with syntax label[guard]/action.
If the action has no parens, then it is interpreted as syntax
sugar for publish(action).  The function "after" is reserved
and must be accompanied by a numeric argument plus units
(e.g., after(3 rti)).
\item
no condition eval w/o event.
\item
if state variables used, they must be declared -- can we spec
semantics
\item
no action from initial state
\end{itemize}

Semantic rules:
\begin{itemize}
\item
no signal-less, condition-less transition out of a state unless the
state is composite and the state has a FinalState.
\item
each statechart (machine) has it's own event management.  That is,
publish() will only be seen in the machine.  There must be another
way to publish events outside of the machine scope.  We need to 
define scopes: m - machine, p - process (address space), c - CPU,
n - network ...
\item
state needs to be managed somehow
\end{itemize}


To add to MagicDraw:
\begin{itemize}
\item
method to tag states as final or progress states
\end{itemize}


Other
\begin{itemize}
\item
can we spec externals?
\item
we need action list because we may flatten machines, and in this case
the exit/enter actions need to be added
\end{itemize}

\section{time}

If we assume a sequence of external events don't occur at the same time
and internally generated events occur instantaneously, we can justify that
when stepping a machine that the internallly generated events should be
applied first.  That is, there is an internal queue and external queue.

Also, time should be attached to each event.



Syntax rules:
\begin{itemize}
\item 
transitions are decorated with syntax label[guard]/action.
If the action has no parens, then it is interpreted as syntax
sugar for publish(action).  The function "after" is reserved
and must be accompanied by a numeric argument plus units
(e.g., after(3 rti)).
\item
no condition eval w/o evnet.
\item
if state variables used, they must be declared -- can we spec semantics
\end{itemize}

Semantic rules:
\begin{itemize}
\item
no signal-less, condition-less transition out of a state unless the
state is composite and the state has a FinalState.
\item
each statechart (machine) has it's own event management.  That is,
publish() will only be seen in the machine.  There must be another
way to publish events outside of the machine scope.  We need to 
define scopes: m - machine, p - process (address space), c - CPU,
n - network ...
\item
state needs to be managed somehow
\item
There may be no action on transition from the initial pseudostate.
\end{itemize}


To add to MagicDraw:
\begin{itemize}
\item
method to tag states as final or progress states
\end{itemize}


Other
\begin{itemize}
\item
can we spec externals?
\end{itemize}


\end{document}
%% --- last line of doc.ltx ---
