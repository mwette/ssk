% ssk_mdrw.tex
%% using magicdraw to generate diagrams.

\iffalse
Copyright (C) 2018 -- Matthew R. Wette.

Permission is granted to copy, distribute and/or modify this document
under the terms of the GNU Free Documentation License, Version 1.3 or
any later version published by the Free Software Foundation; with no
Invariant Sections, no Front-Cover Texts, and no Back-Cover Texts.  A
copy of the license is included with the distribution as @file{COPYING.DOC}.
\fi



\documentclass{article}
\usepackage{times}
\usepackage{amsmath}
%\usepackage[frame,matrix]{xy}
\usepackage[all]{xy}

\setlength{\textheight}{9in}
\setlength{\textwidth}{6.5in}
\setlength{\topmargin}{-.5in}
\setlength{\evensidemargin}{0in}
\setlength{\oddsidemargin}{0in}

\title{%
Use of MagicDraw for Generating StateCharts
}
\author{Matt Wette \\ project }
\date{\large DRAFT of \noindent\today \ /\the\time}

\begin{document}
\maketitle

\section{Guildlines}

old notes:
\begin{verbatim}
transitions:
  trigger -> signalevent -> signal -> name
  effect ->
    activity -> name

HACKS:

if effect is opaquebehavior use name for action expression

if effect is functionbehavior name is function name (no args), use body

in case where there is specification->body and name use body first, then name
\end{verbatim}

Declare signal triggers at global level:
\begin{enumerate}
\item
Right click on the ``Data'' field of the Containment window.
\item
Click on ``New Element ...'', select ``Package'' and name it ``Events.''
\item
Right click on the ``Data'' field of the Containment window.
\item
Click on ``New Element ...'', select ``Package'' and name it ``Actions.''
\end{enumerate}

Generating Transitions:
\begin{enumerate}
\item
Please declare events using ``Signal Event''.  Use ``Signal'' type
and select a signal you have declared as above.
\item
For guards just use the guard field.
\item 
For actions use ``Function Behavior'' and provide a name for the action.
You should handle what the name does later.   Note that actions cannot
take arguments.  NEED TO LOOK INTO THIS.
\end{enumerate}

\section{Stuff}

For all signals, in ``containment'' area create signal (right-click on
project and scroll down to signal ...

effects (aka actions): Options:
\begin{enumerate}
\item
Use OpaqueBehavior.  But only the name shows up on the diagram.
\end{enumerate}

Suggest use \texttt{send\_SignalName} or \texttt{send2\_SignalName}
to send to port 2.


\section{Miscellaneous Comments}

The UML2 spec use of isComposite and isOrthogonal is lame.  Why not just
say ``type'' is Simple, Composite, Orthongal, or Submachine?  So much easier.
Instead there are four member variables: isSimple, isComposite, isOrthogonal
and isSubmachine.  The only time more than one is true is when we have an
orthognal state in which case isComposite is also true.

If you get something wrong in MagicDraw it may take some work to fix.  
For example, if you have a trigger on a transition from an initial state
the SSK will tag this as an error.  To remove the trigger you need to remove
it from the data hierarchy: I don't think you can remove in the trigger dialog.

\section{diagrams}

\begin{displaymath}
\UseTips
\entrymodifiers={++[o][F]}
\xymatrix @-1mm {
 *+\txt{in} \ar[r] 
& 1 \ar@(dr,dl)[]^b \ar[r]_a
& 2 \ar@(d,dl)[]^a \ar[r]_b
& 3 \ar `u[l] `^d[l]_a [l] \ar[r]_b
& *++[o][F=]{4}
    \ar `dl_l[ll]+/d6mm/`l_ul[ll]^a [ll]
    \ar `u^l[lll]+/u1cm/`l^d[lll]_b [lll]
}
\end{displaymath}

\end{document}
%% --- last line of ssk_mdrw.tex ---
